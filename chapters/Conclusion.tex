\chapter{おわりに}\label{cha:Conclusion}
本論文では、ゲーム開発におけるテスト効率の向上を目的として、テスト支援ツールを試作した。
テスト支援ツールの、オブジェクトとプレイヤーの重なり判定および自動リプレイ機能を利用することによって
ゲーム制作時の支援を行う。テスト支援ツールは、次に示す10個の機能を持つ。

\begin{itemize}
	\item キャラクター情報入力機能
	\item オブジェクト増減機能
	\item オブジェクト情報入力機能
	\item オブジェクト位置確認機能
	\item ゲームスタートボタン機能
	\item リプレイボタン機能
	\item Warning判定出力機能
	\item 入力キーとタイミングの記録機能
	\item リプレイ機能
	\item エラーメッセージ出力機能
\end{itemize}

本論文で試作したテスト支援ツールは、5章の適用例で示したように正しく動作することが確認できた。

本論文で試作したテスト支援ツールは、2Dゲームのプレイヤーと四角形のオブジェクトの重なり判定を行い、自動でファイルに出力できる。
また、1度プレイしたゲームを自動でリプレイできる。

テスト支援ツールを使用しない場合は、チーム開発の際のテストの報告時に報告書や口頭による説明では人為的なミスが生じてしまう可能性がある。
また、1度行ったテストを再現するために同じようにゲームをプレイするのは、手前と時間がかかる。

テスト支援ツールを使用した場合は、オブジェクトの重なり判定を自動で行い、自動でファイルに報告書として保存する。そのため、チームへのテスト結果を共有する際に、人為的なミスが生じる可能性を減らすことができる。
それに加え、入力履歴を保存しているリプレイファイルをツールに読み込むことにより、自動で何度もゲームのリプレイが可能である。それにより、同じ条件にするために何度もプレイする手間がかからないので、時間の削減になる。

以上のことから、試作したテスト支援ツールを利用することによって、ゲーム制作時の支援を行うことができると言える。


以下に、テスト支援ツールの今後の課題を示す。

\begin{itemize}
	\item 実際に動かしたゲームとリプレイに誤差への対応\\		
		現段階では、ゲームをプレイする時とリプレイする時では、キャラクターの動きに誤差が生じてしまう(図5.15、5.16参照)。
		誤差を少なくする方法としては、ゲームをプレイする際のキー入力の取得処理を並列実行することにより、誤差を減らすことが可能と考える。
		
	\item オブジェクト毎の確認画面の実装\\
		現段階では、オブジェクト全体の確認画面はあるが、1つ1つのオブジェクトに対して確認する画面が未実装である。
		オブジェクト情報入力ウィンドウに新しくボタンを追加し、4.4節の実装と同様の実装を行うことによって、実現可能と考える。
	
	\item Warningの種類を増やす\\
		現段階では、Warningとして出力する対象は、キャラクターと動かない四角形のオブジェクトの重なり判定のみである。
		キャラクターと動くオブジェクトの重なり判定や、キャラクターの画像の種類の判別などは、現在のテスト支援ツールでは対応できていない。
		例えばキャラクターと動くオブジェクトの重なり判定をWarningとして出力する場合は、パーティクルフィルターを複数対応させることによって、
		実現可能と考える。
		
	\item 速度の速いプレイヤーを検出する\\
		現段階では、キャラクターの移動速度が速すぎるとパーティクルがキャラクターを検出するまでに時間がかかってしまう。
		検出時間を短縮するためには、現在300個に設定しているパーティクル数を増やすことによって、実現可能だと考える。
			
	\item 複数キー入力に対応する\\
		現段階では、入力ボタンとタイミングの記録機能およびリプレイ機能において、複数キー入力に対応できていない。
		複数キー入力に対応するためには、入力ボタンとタイミングの記録機能で複数キーの取得を行い、リプレイ機能もそれに応じた改良を行うことによって、実現可能だと考える。
		
\end{itemize}




