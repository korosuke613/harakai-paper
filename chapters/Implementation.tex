\chapter{実装}\label{cha:Implementation}

本章では、本研究で試作したモータ特性表自動生成ツールの実装について説明する。

\section{特性表生成機能}\label{tokuseihyou_seisei}

% モータ特性表自動生成ツールの処理の流れを図\ref{fig:syori}に示す。\\

% \begin{figure}[t]
% 	\centering
% 	\includegraphics[width=16.5cm,height=10cm]{./Image/sample.png}
% 	\caption{モータ特性表自動生成ツールの処理の流れ}
% 	\label{fig:syori}
%   \end{figure}

特性表生成機能の処理の流れを以下に示す。
\begin{enumerate}
    \item OpenModelicaから出力されたcsvファイルを読み込む
    \item 特性表の各要素を算出するために必要なデータをcsvファイルから取得する
    \item 特性表の各要素を算出する
    \item 特性表を生成する
\end{enumerate}

以下、各処理について具体的に説明する。

\subsection{csvファイルの読み込み}\label{sub:csvfairu}
Pythonで実装するため、Pythonの標準ライブラリのcsvモジュールをインポートし、
csvファイルを読み込む。

\subsection{特性表の各要素を算出するために必要なデータを取得}\label{sub:syutoku_data}
\ref{sub:csvfairu}章で読み込んだcsvファイルから、以下のデータを取得する。

\begin{itemize}
    \item 時間
    \item 電流
    \item 電圧
    \item トルク
    \item 角速度
\end{itemize}

取得方法としては、まず、図\ref{fig:simyu_csv}にあるように、OpenModelicaから出力されたcsvファイルの1行目には
各部品の変数名が記載されているので、取得したいデータを持つ変数名を探し、その変数名がある列番号を取得する。
そして、各データに対応した配列に、列番号の位置にある値を繰り返し処理で格納する。

\subsection{特性表の各要素を算出}\label{sub:keisan}
\ref{sub:syutoku_data}章で取得したデータを用いて、\ref{kenkyu_mokuteki}章で挙げた各要素を算出する。


% \item 始動電流 mA
% \item 停動トルク mNm
% \item 最大効率 \%
% \item 定格トルク mNm 
% \item 定格回転数 rpm
% \item 定格電流 mA
% \item 定格出力 W
% \item 定格電圧 V
% \item 最大回転数 rpm 

\subsection{始動電流}\label{sub:sidouden}
始動電流とは
 
\subsection{停動トルク}\label{sub:teidoutoruku}


\subsection{最大効率}\label{sub:saidaikouritu}


\subsection{定格トルク}\label{sub:teikakutoruku}


\subsection{定格回転数}\label{sub:teikakukaiten}


\subsection{定格電流}\label{sub:teikakuden}


\subsection{定格出力}\label{sub:teikakusyutu}


\subsection{定格電圧}\label{sub:teikakudennatu}


\subsection{最大回転数}\label{sub:saidaikaiten}




\subsection{特性表を生成}\label{sub:seisei_hyou}
