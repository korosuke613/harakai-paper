\chapter{機能}\label{cha:ExistingRETUSS}

この章では、テスト支援ツールの概観とその機能について説明する。テスト支援ツールは、以下の2つの特徴をもつ。
\begin{itemize}
	\item キャラクターとオブジェクトの重なり判定を自動で行い、報告書としてファイルに出力
	\item プレイしたゲームの自動リプレイ
\end{itemize}

なお、今回対象とするゲームは、Unityで制作した1画面のみで完結する2Dゲームで、同時に複数のキーを入力しないものとする。今回対象とするキーは「Aキー、Sキー、Wキー、Dキー、スペースキー、左矢印キー、上矢印キー、右矢印キー、下矢印キー」である。
また、実装の都合上、今回設定できるオブジェクトの数を4つまでとする。動作環境はWindowsで、設定できるキャラクターの画像の拡張子は、pngかjpegとする。
なお、設定するキャラクターの画像に似た画像を使用するゲームの場合は、今回は対象外とする。これは、
重なり判定にパーティクルフィルターを用いることが原因で、誤
判定が生じてしまうためである。


\section{対応するモデル}\label{pre_kinou}
テスト支援ツールの概観を、図3.1に示す。テスト支援ツールは、以下のエリアからなる。\\
\begin{itemize}
	\item キャラクター情報設定部
	\item オブジェクト追加部
	\item データ入力部
	\item 実行部
	\item エラーメッセージ出力表示部
\end{itemize}

% \begin{figure}
%   \centering
%   \includegraphics[width=15cm]{./Image/3.1.eps}
%   \caption{テスト支援ツールの概観}
% \end{figure}

\subsection{モータ単体のモデル}\label{sec:sub1}
キャラクター情報設定部は、重なり判定させたいキャラクター名の表示と、そのキャラクターのキャラクター情報入力ウィンドウを新しく生成し、表示する。
キャラクター情報設定部は、以下の2つの要素をもつ。キャラクター情報設定ボタンを押下すると、3.2.1節に示すキャラクター情報入力機能を利用できる。
\begin{itemize}
	\item キャラクター名
	\item キャラクター情報設定ボタン
\end{itemize}

\subsection{パッケージ化されたモデル}\label{sec:sub2}
キャラクター情報入力ウィンドウの概観を、図3.2に示す。キャラクター情報入力ウィンドウはキャラクター情報入力部と、
キャラクター情報エラーメッセージ出力表示部で構成している。
キャラクター情報入力部は、ゲームで追跡させたいキャラクターの画像と名前を入力できる。
キャラクター情報エラーメッセージ出力表示部は、キャラクター情報入力部で設定していない情報があった場合、エラーメッセージを表示する。
キャラクター情報入力部は、以下の3つの要素をもつ。

% \begin{itemize}
% 	\item キャラクター画像
% 	\item キャラクター名入力テキストボックス
% 	\item キャラクター情報保存ボタン
% \end{itemize}

% % \begin{figure}
% %   \centering
% %   \includegraphics[width=10cm]{./Image/3.2.eps}
% %   \caption{キャラクター情報入力ウィンドウの概観}
% % \end{figure}

% \subsection{オブジェクト追加部}\label{sec:sub3}
% オブジェクト追加部は、キャラクター情報設定部で設定したキャラクターと重なり判定を行う範囲を指定するオブジェクト情報入力ウィンドウを新しく生成し、表示する。
% また、四角形のオブジェクトのゲーム画面上での位置を確認できる。さらに、オブジェクト位置確認ウィンドウを新しく生成し、表示する。
% オブジェクト追加部は、以下の4つの要素をもつ。

% \begin{itemize}
% 	\item オブジェクト名
% 	\item オブジェクト情報設定ボタン
% 	\item オブジェクト増減ボタン
% 	\item オブジェクト位置確認ボタン
% \end{itemize}


% \subsection{オブジェクト情報入力ウィンドウ}\label{sec:sub4}
% オブジェクト情報入力ウィンドウの概観を、図3.3に示す。オブジェクト情報入力ウィンドウは、オブジェクト情報入力部と、
% オブジェクト情報エラーメッセージ出力表示部で構成している。
% オブジェクト情報入力部は、四角形のオブジェクトの座標の範囲を指定できる。
% オブジェクト情報エラーメッセージ出力表示部は、オブジェクト情報入力部で設定していない情報があった場合、エラーメッセージを表示する。
% オブジェクト情報入力部は、以下の3つの要素をもつ。

% \begin{itemize}
% 	\item オブジェクト座標入力テキストボックス
% 	\item オブジェクト名入力テキストボックス
% 	\item オブジェクト情報保存ボタン
% \end{itemize}

% % \begin{figure}
% %   \centering
% %   \includegraphics[width=10cm]{./Image/3.3.eps}
% %   \caption{オブジェクト情報入力ウィンドウの概観}
% % \end{figure}

% \subsection{オブジェクト位置確認ウィンドウ}\label{sec:sub100}
% オブジェクト位置確認ウィンドウの概観を、図3.4に示す。オブジェクト位置確認ウィンドウは、ゲームの背景画像と、背景変更ボタンで構成している。
% 背景変更ボタンを押下して、画像を選択すると、選択した画像をゲームの背景画像が表示できる。
% この時に、オブジェクト追加部で四角形のオブジェクトの範囲を指定していた場合、ゲームの背景画面の上に、オブジェクトの範囲を表示できる。
% こうすることで、ゲームの背景画像に対して重なり判定させたいオブジェクトの位置を、正しく指定できていることを確認できる。

% % \begin{figure}
% %   \centering
% %   \includegraphics[width=10cm]{./Image/3.4.eps}
% %   \caption{オブジェクト位置確認ウィンドウの概観}
% % \end{figure}

% \subsection{データ入力部}\label{sec:sub3}
% データ入力部では、以下の3つの情報を入力できる。
% \begin{itemize}
% 	\item 報告者の名前
% 	\item 実行するゲームの実行ファイルのパス(.exe)
% 	\item ゲーム実行時に入力したボタンの履歴を記録した出力ファイルのパス(.csv)
% \end{itemize}


% \subsection{実行部}\label{sec:sub5}
% ユーザーは、実行部に配置したボタンを選択し、押下することによって、選択したテスト支援ツールの機能を使用できる。実行部のボタンを、以下に示す。\\
% \begin{itemize}
% 	\item ゲームスタートボタン
% 	\item リプレイボタン
% \end{itemize}


% \subsection{エラーメッセージ出力表示部}\label{sec:sub6}
% エラーメッセージ出力表示部は、実行部のゲームスタートボタンや、リプレイボタンを押下した際に、
% キャラクター情報設定部、オブジェクト追加部、データ入力部で設定していない情報があった場合に、エラーメッセージを表示する。

\section{特性表生成}\label{kenkyu_mokuteki}

% テスト支援ツールは、次に示す10個の機能を持つ。

% \begin{itemize}
% 	\item キャラクター情報入力機能
% 	\item オブジェクト増減機能
% 	\item オブジェクト情報入力機能
% 	\item オブジェクト位置確認機能
% 	\item ゲームスタートボタン機能
% 	\item リプレイボタン機能
% 	\item Warning判定出力機能
% 	\item 入力キーとタイミングの記録機能
% 	\item リプレイ機能
% 	\item エラーメッセージ出力機能
% \end{itemize}

% \subsection{キャラクター情報入力機能}\label{sec:sub5}
% キャラクター情報入力機能は、キャラクター情報設定部に配置しているキャラクター情報設定ボタンを押下することによって利用できる。
% キャラクター情報設定ボタンを押下すると、キャラクター情報入力ウィンドウを新しく生成して表示し、ユーザーはキャラクター情報を入力できる(図3.2参照)。

% \subsubsection{キャラクター画像の選択}
% ユーザーは、キャラクター情報入力部のキャラクター画像選択ボタンを押下すると、Windows標準のファイル選択ダイアログが開き、画像ファイルを選択できる。
% 選択した画像を、キャラクター画像選択ボタンに表示する。

% \subsubsection{キャラクター名の入力}
% ユーザーは、キャラクター情報入力部のキャラクター名入力テキストボックスを押下することによって、キャラクター名を入力できる。

% \subsubsection{キャラクター情報の保存}
% ユーザーは、キャラクター情報保存ボタンを押下することによって、選択したキャラクター画像と、入力したキャラクター名を保存できる。

% \subsection{オブジェクト増減機能}\label{sec:sub11}
% オブジェクト追加部のオブジェクト増減ボタンを押下すると、四角形のオブジェクトを追加または削除できる。
% オブジェクト増減ボタンを押下することによって、オブジェクト名とオブジェクト情報設定ボタンをオブジェクト追加部に追加する。
% オブジェクト名は、オブジェクト情報入力部で入力した名前を表示する。

% \subsection{オブジェクト情報入力機能}\label{sec:sub5}
% オブジェクト情報入力機能は、オブジェクト追加部に配置しているオブジェクト情報設定ボタンから利用できる。
% オブジェクト情報設定ボタンを押下すると、オブジェクト情報入力ウィンドウを表示でき、ユーザーは四角形のオブジェクト情報を入力できる(図3.3参照)。

% \subsubsection{オブジェクトの座標の設定}
% ユーザーは、キャラクター情報設定部で設定した、キャラクターとの重なり判定を調べたいオブジェクトのゲーム画面上での座標を設定できる。
% オブジェクトの座標は、オブジェクト情報入力部にあるオブジェクト座標入力テキストボックスに入力する。

% \subsubsection{オブジェクト名の入力}
% ユーザーは、オブジェクト情報入力部のオブジェクト名入力テキストボックスに、オブジェクト名を入力できる。

% \subsubsection{オブジェクト情報の保存}
% ユーザーは、オブジェクト情報入力部のオブジェクト情報保存ボタンを押下することによって、入力したオブジェクトの座標と入力したオブジェクト名を保存できる。

% \subsection{オブジェクト位置確認機能}
% ユーザーは、オブジェクト追加部で追加した、ゲーム画面上でのオブジェクトの範囲をオブジェクト位置確認ウィンドウ(図3.4参照)で確認できる。
% 1度背景を登録すると、2回目にオブジェクト位置確認ボタンを押下して、オブジェクト位置確認ウィンドウを開いた場合でも、1回目に登録したゲーム背景画像を表示する。このため、
% 何度も同じ背景を登録し直すことなく、オブジェクトの位置を確認できる。

% \subsection{ゲームスタートボタン機能}
% 実行部に配置しているゲームスタートボタンを押下すると、データ入力部で設定したゲームを実行する。
% ゲームがスタートすると、入力ボタンとタイミングの記録と、キャラクター情報入力ウィンドウのキャラクター情報入力部で
% 設定したキャラクターの追跡、 Warning判定出力を行う。

% \subsection{リプレイボタン機能}
% リプレイボタンを押下すると、データ入力部で設定したゲームを実行する。ゲームがスタートすると、リプレイを行う。

% \subsection{Warning判定出力機能}\label{sec:sub7}
% ゲーム終了時に、キャラクター名とキャラクターに重なったオブジェクト名、および、時刻と報告者をファイルに出力する。
% このファイルを、以後報告書と定義する。

% \subsection{入力キーとタイミングの記録機能}\label{sec:sub8}
% ゲーム終了時に、ゲーム実行中に入力したキーと、ゲーム実行時からキーを入力するまでの時刻を記録し、ファイルに出力する。
% このファイルを、以後リプレイファイルと定義する。

% \subsection{リプレイ機能}\label{sec:sub9}
% データ入力部で設定したリプレイファイルを読み込み、
% 記録している情報通りに、ゲームのキャラクターを自動で動かす。

% \subsection{各機能のエラーメッセージの出力機能}\label{sec:sub10}
% エラーメッセージ出力表示部には、各機能のエラーメッセージの出力を行う。具体的には、以下の場合にエラーとして出力する。

% \begin{itemize}
% 	\item ゲームスタートボタンを押下した際に、実行するゲームファイルや報告者を設定していない場合
% 	\item リプレイボタンを押下した際に、実行するゲームファイルや入力したキーとその時刻が記録しているリプレイファイルを設定していない場合
% 	\item オブジェクト追加部で5つ以上のオブジェクトを追加した場合
% \end{itemize}


%\input{CommonTexs/Items_Area}
%以降、それぞれの処理部について説明する。



