\chapter{考察}\label{cha:Discussion}
本論文では、モータ特性表自動生成ツールを試作した。

\section{評価}

\subsection{評価方法}
人手によるドメイン分析テストのためのテストケース作成と、拡張したBWDMによるドメイン分析テストのためのテストケース生成で、作成(生成)に要した時間の比較検証を行った。
その結果を、表\ref{tab:time}に示す。

対象としたVDM++仕様は、\ref{cha:domain}節で用いた コード\ref{fig:vdm_park}である。
``割引価格となる''を期待出力に持つドメインに対するドメイン分析テストのためのテストケースを作成する時間を計測した。
生成するテストケースとしては、以下を基準とした。
\begin{enumerate}
  \item onポイント、offポイント、inポイント、outポイントを出力(記述)する
  \item onポイント、offポイント、outポイントには、着目条件式も出力(記述)する
  \item offポイントには、着目変数も出力(記述)する
  \item 各ポイントには、期待出力と正常系であるかどうかも出力(記述)する
\end{enumerate}

検証に参加したメンバーは本研究室の大学院生3人と学部4年生1人であり、
普段からソースコードの読み書きを行い、基本的なプログラミングの知識を有している。
VDM++の文法の知識を持たない者も含まれるが、
今回の検証に必要な文法は、事前に他のVDM++の例を用いてレクチャーした。
また、ドメイン分析テストのためのテストケース生成についても、事前に他のVDM++仕様とテストケースの例を用いてレクチャーした。

人手による検証では、
コード\ref{fig:vdm_park}を印刷した紙を渡し、
仕様を確認後、
テストケースを書き始めてから、テストケースを記述し終えるのに要した時間を計測した。
入力データと戻り値の組合せが不正確な場合、間違いを指摘し、
被験者が正しい組合せを記述した時点で時間計測終了とした。
また、制限時間を30分とし、制限時間を超えた場合、その場で時間計測終了とした。

拡張したBWDMによる検証では、
コマンドライン上での命令操作で、拡張したBWDMによるテストケース生成を行うのに要した時間を計測した。
また、実験に用いたコンピュータは、OS:macOS 10.14.5、CPU:2.3GHz Intel Core i5、メモリ:16GBである。

なお、JavaのSystem.nanoTime\cite{nanotime}メソッドを用いて、
命令操作を省いた純粋なテストケース生成処理にBWDMが要した時間を計測した結果、
1.25秒であった。

人手による作成と比較した結果、平均で18分程の時間短縮を確認できた。
対象にしたVDM++仕様には、VDM++独特の文法等は含まれないため、
VDM++に対する慣れなどの影響は無視できるものと思われる。
また、人手によるテストケース生成の場合、ヒューマンエラーも見られた。
具体的には、offポイントの記述時に、条件式の解釈を間違え、誤った期待出力を記述してしまった。(例:入力(17、 20)の期待出力を``遊園地チケットは割引価格とならない。(妻の年齢 $<$ 16)''と記述した。)
仕様の規模が拡大すると、人手とコンピュータとの処理効率の差に加えて、
ヒューマンエラーの有無などにより、テストケース生成に要する時間の差は更に拡大していくと思われる。
以上から、拡張したBWDMは有用性が向上したと考える。

\begin{table}[tp]
\centering
\caption{コード\ref{fig:vdm_park}のドメイン分析テストのためのテストケース作成に要した時間の比較}
\label{tab:time}
\begin{tabular}{cc}
\begin{minipage}[c]{0.5\hsize}
  \centering
  \begin{tabular}{c|c}
    被験者  & 時間              \\
    \hline
    \hline
    被験者A & 8m 16s            \\ \hline
    被験者B & 10m 23s           \\ \hline
    被験者C & 30m(制限時間超過) \\ \hline
    被験者D & 24m 04s
  \end{tabular}
\end{minipage} &
\begin{minipage}[c]{0.5\hsize}
  \centering
  \begin{tabular}{c|c}
                 & 時間    \\
    \hline
    \hline
    被験者(平均) & 18m 10s \\ \hline
    BWDM         & 0m 15s
  \end{tabular}
\end{minipage}
\end {tabular}
\end{table}



\subsection{結果}
本論文で試作したモータ特性表自動生成ツールは、

\section{関連研究}

	関連研究について述べる。

\section{ツールの問題点}

以下に、今回作成したモータ特性表自動生成ツールの問題点を示す。

\begin{itemize}
	\item 対応するモータのモデルは1種類しかない\\
		  モータは~種類に分けることができ、今回は1つにしか対応していない。
		  対応できる数を増やす必要がある。
		
\end{itemize}







