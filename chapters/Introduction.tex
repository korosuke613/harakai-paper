\chapter{はじめに}\label{cha:Introduction}


ものづくりにおいて、シミュレーションを行うことは重要である。[参考文献]製品を開発する際にシミュレーション行うことで、実際に製品を試作することなく、製品の性能を確認できるため、コストを削減できる。[参考文献]また、製品開発の納期短縮できる。[参考文献]

シミュレーションツールの1つにOpenModelicaがある。OpenModelicaでは、~を対象として~をすることができる。しかし、OpenModelicaが出力する結果は、グラフや数値であるため、特定の値を確認するのに時間が掛かるという問題がある。この問題を解決する手段の1つとして、特性表を用いることが考えられる。特性表とは、製品の性能をまとめた表である。しかし、特性表の作成は、人手により作成するため、手間と時間を必要とする。OpenModelicaのシミュレーション結果を用いて、特性表を作成するためには、OpenModelicaが出力したcsvファイルから特定の値を算出し、グラフを生成する必要がある。そこで、本研究では、特性表を作成するためにかかる時間の削減を目的として、OpenModelicaのシミュレーション結果を用いた特性表自動生成ツールの試作を行う。なお、本研究では、ブラシ付きDCモータ[参考文献]を対象とする。

本論文の構成は、以下の通りである。\\
第2章では、モータ特性表自動生成ツールを試作するために必要となる前提知識について説明する。\\
第3章では、試作したモータ特性表自動生成ツールの構成及び手順について説明する。\\
第4章では、試作したモータ特性表自動生成ツールが正しく動作することを検証する。\\
第5章では、試作したモータ特性表自動生成ツールについて考察する。\\
第6章では、、本論文のまとめとの課題を述べる。\\