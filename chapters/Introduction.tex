\chapter{はじめに}\label{cha:Introduction}
近年、モータは、エアコン・洗濯機・掃除機などの家電製品をはじめ、自動車関係、医療関係など様々な分野に
用いられており\cite{モータ使用製品}、社会に必要不可欠な存在となっている。\\
~はじめに 流れ 案~\\
モータの開発は~で、~の課題がある。それを解決する手段としてシミュレーションがある。
シミュレーションツールの中にOpenModelicaがある。OpenModelicaは~で、~する。
また、結果を画面所にプロットすることで結果を確認できる。
シミュレーションを行った場合、期待通りか結果と比較する。\\
比較する際は、シミュレーション結果から目的のグラフや値を計算等して作成しなければならない。\\
しかし、OpenModelicaではグラフでしか確認できず、具体的な値を取得することが困難である。
そこで、本研究では、モータのシミュレーションをOpenModelicaで行った際のシミュレーション結果の確認にかかる手間を削減することを目的として、
シミュレーション結果のcsvファイルからモータ特性表を自動生成するツールを試作する。\\
% 今回試作したツールで、グラフや値を作成する手間を省くことで、モータ開発の効率化を図る。\\



本論文の構成は、以下の通りである。\\
第2章では、モータ特性表自動生成ツールを試作するために必要となる前提知識について説明する。\\
第3章では、試作したモータ特性表自動生成ツールの構成及び手順について説明する。\\
第4章では、試作したモータ特性表自動生成ツールが正しく動作することを検証する。\\
第5章では、試作したモータ特性表自動生成ツールについて考察する。\\
第6章では、、本論文のまとめとの課題を述べる。\\