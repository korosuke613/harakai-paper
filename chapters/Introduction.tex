\chapter{はじめに}\label{cha:Introduction}
ものづくりにおいて、実際に製品を作る前にシミュレーションを行うことは重要である[参考文献]。シミュレーションを行うことにより、実際に製品を試作することなく、製品の様々な性能を事前に確認できるため、開発コストを削減できる[参考文献]。また、試作品を実際に作る回数を減らすことができるため、製品開発の納期の短縮が期待できる。

シミュレーションツールの1つにOpenModelicaがある。OpenModelicaでは、~を対象として~をすることができる。OpenModelicaが出力するシミュレーションの結果は、グラフや数値であり、CSVファイルに出力することができる。しかし、この出力を用いて、性能を決定付ける特定の値を確認するためには、手間と時間がかかる。そこで、本研究では、特定の値を確認するためにかかる時間の削減を目的として、OpenModelicaのシミュレーション結果を用いた特性表の自動生成ツールの試作を行う。特性表とは、製品の性能をまとめた表であり、特性表を用いることで、特定の値を容易に確認できる。なお、本研究では、シミュレーションの対象として、ブラシ付きDCモータ[参考文献]を対象とする。


% この問題を解決する手段の1つとして、特性表を用いることが考えられる。しかし、特性表の作成は、人手により作成するため、手間と時間を必要とする。OpenModelicaのシミュレーション結果を用いて、特性表を作成するためには、OpenModelicaが出力したcsvファイルから特定の値を算出し、グラフを生成する必要がある。

本論文の構成は、以下の通りである。\\
第2章では、モータ特性表自動生成ツールを試作するために必要となる前提知識について説明する。\\
第3章では、試作したモータ特性表自動生成ツールの構成および手順について説明する。\\
第4章では、試作したモータ特性表自動生成ツールが正しく動作することを検証する。\\
第5章では、試作したモータ特性表自動生成ツールについて考察する。\\
第6章では、、本論文のまとめとの課題を述べる。\\